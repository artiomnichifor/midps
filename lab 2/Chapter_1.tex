\section{Desfasurarea lucrarii de laborator}

\subsection{Conditii si cerinte}

Basic Level (nota 5 || 6):

Realizeaza un simplu GUI calculator care suporta functiile de baza: +, -, /, *.
Normal Level (nota 7 || 8):

Realizeaza un simplu GUI calculator care suporta urmatoare functii: +, -, /, *, putere, radical, InversareSemn(+/-).
Advanced Level (nota 9 || 10):

Realizeaza un simplu GUI calculator care suporta urmatoare functii: +, -, /, *, putere, radical, InversareSemn(+/-), operatii cu numere zecimale.
Divizare proiectului in doua module - Interfata grafica(Modul GUI) si Modulul de baza(Core Module).

\subsection{Analiza lucrarii de laborator}
	Linkul repozitoriului  \url{https://github.com/artiomnichifor/midps}

	Esenta lucrarii date de laborator a constat in acomodarea cu mediul de lucru al GUI si studiul lor. Scopul lucrarii date a fost de a crea o simpla interfata grafica care poseda proprietatile unui calculator. Am studiat metoda de creare a unei ferestre cit si elementele acesteiadestinate interactiunii cu utilizatorul si prelucrarii operatiilor aritmetice. Folosin IDE Visual Studio am avut posibilitatea cu usurinta sa definim interfata grafica studiind logica crearii si reprezentarii acesteia pentru o comunicare flexibila cu utilizatorul cit si formarea unei logici structurate pentru rezolvarea algoritmilor. Creind fereastra am inceput cu definirea proprietatilor generale ale acesteia ca denumirea, dimensiunele statice, pozitiia pe desktop la aparitie, apoi am creat elementele ferestrei cu marimele, fonturile, culorile si denumirile sale. Urmatorul pas a constatat in scrierea codului pentru functiile fiecarui buton, input- ul si output- ul acestora, si realatiile dintre butoane in crearea si afisarea rezultatului in textbox. Am defint un label pentru afisarea informatiei intermediare despre efectuarea operatiei si am facut posibil ca datele sa fie inscrise si de la tastatura legind fiecare cifra si operand cu butonul sau. Am constatat empiric eficacitatea si corectitudinea programului pentru calcul si cu numere intregi. In final am obtinut un program ce poate efectua simple operatii oferind interfata usor de perceput de catre utilizator bazata pe caracterul uzual al
calculatorului.





\subsection{Imagini}

\includegraphics[]{C:/Users/nichi/Desktop/midps lab 2/calc.PNG}


	GUI Calculator


\includegraphics[]{C:/Users/nichi/Desktop/midps lab 2/clac label.PNG}

	Exemplu de utilizare a label- ului pentru afisarea datelor intermediare




