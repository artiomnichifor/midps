\section{Desfasurarea lucrarii de laborator}

Se considera ca ai trecut cu succes laboratorul daca ai urmat toti pasii din Submission Process:
\begin{itemize}
	\item Lucreaza la proiect in echipa de 2-3 persoane
	\item Divizeaza task-urile si descrie-le in raport, indicind pentru fiecare cine este responsabil pentru el.
	\item Fiecare din membru va avea propriul raport care va include propriile observatii si concluzii.
	\item Fiecare din membrii echipei va lucra pe propriul branch in git, iar una din persoane va avea grija sa faca merge cu master.
	\item Proiectul se poate afla doar in repozitoriul unui membru al echipei.
\end{itemize}

\subsection{Analiza lucrarii de laborator}

	Linkul repozitoriului  \url{https://github.com/artiomnichifor/midps}

	Scopul lucrarii date a fost cel de a forma deprinderea lucrului in grup a studentilor. Lucrul in echipa reprezinta una din practicele esentiale ale activitatii unui inginer in domeniul IT.

	Esenta lucrarrii date a constat in formarea unoei aplicatii "Contact List" pe platforma Android si repartizarea taskurilor intre membrii echpei. Etapele de executie au fost: 
\begin{itemize}
	\item Formarea corpului de baza a programului bazat pe layer cu intentii si widgeturi
	\item Crearea unei interfete precum si oformarea paginii
	\item Crearea unei baze de date
	\item Formarea raportului si a unei concluzii adecvate
\end{itemize}

	 Formarea layerului a stat la baza intregului program, care a constat din crearea unui meniu pentru repartizarea listei de contacte de partea destinata crearii acesteia, butoanelor si textboxurilor pentru memorarea si afisarea informatiei. Prelucrea eventurilor a avut loc datorita intentiilor si clasei ViewList creata cu scopul de a afisa  in lista informatia despre fiecare contact. La rindul sau o buna oformare a paginii a fost cauzata de necesitatea unei interfete placute si usor de inteles. 


\begin{figure}[h!]
			\centering
 			 \includegraphics[scale=1]{"firstpage"}
 			 \caption{Adaugarea unui contact in lista}
 			 \label{fig:Pagina principala}
		\end{figure}	
Informatia despre fiecare contact poate fi vuzializata in cadrul unei liste
	
	\begin{figure}[h!]
			\centering
 			 \includegraphics[scale=1]{"2nd page"}
 			 \caption{Lista de contacte}
 			 \label{fig:AnimePage}
		\end{figure}



In cadrul lucrului in echipa, fiecare membru a muncit pe un branch iar la sfarsit am realizat merge pe branch-ul master.

	\begin{figure}[h!]
			\centering
 			 \includegraphics[scale=1]{"merge"}
 			 \caption{Utilizarea comenzii merge}
 			 \label{fig:Calcui}
		\end{figure}











