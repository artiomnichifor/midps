\section*{Concluzie}
\phantomsection
 	
	Esenta lucrarii date de laborator a constat in acomodarea cu principiile de dezvoltare si functionare a unei aplicatii pe platforma Android. Android reprezinta un sistem de operare dominant pe piata software pentru dispozitive mobile, care este si foarte efectiv permitind crearea unei game largi de aplicati cu destinatii diferite. Android Studio, IDE -ul pe care a fost efectuata lucrarea data, reprezinta mediul integrat cu facilitati specializate in formarea interactiva a unei aplicatii. Desi destul de costisitoare, utilizind si mult timp pentru activarea emulatorului sau instalarea aplicatiei pe dispozitiv, Android Studio functioneaza dupa principiile generale de simplificare a etapelor de constructie a codului si interfetei, oferind o metoda relativ simplificata de formare a aplicatiei. Este destul de dificil la prima vedere de perceput particularitatile acestuia, dar datorita sablonarii si unei documentatii bogat erorile sunt usor de depasit. Un alt plus major fiind sistemul de debbuging, a interfetii de lucru cit si cel de completare a comenzilor, prezente intr -un IDE decent si usor de studiat. In urma lucrarii de laborator am sistematizat proprietatile interactiuniilor dintre elementele aplicatiei, UI, Layouts, Widgets etc. si am format cunostinte de baza despre platoforma SDK.

\clearpage