\section{Desfasurarea lucrarii de laborator}
Obiective:
\begin{itemize}
	\item Cunostinte de baza privina arhitectura unei aplicatii mobile
	\item Cunostinte de baza ale platformei SDK
\end{itemize}

Se considera ca ai trecut cu succes laboratorul daca ai urmat toti pasii din Submission Process:
\begin{itemize}
	\item Trebuie sa elaborezi un program prototip care il vei arata in timpul laboratorului
	\item Ai respectat DL (data limita)
\end{itemize}
\subsection{Conditii si cerinte}

Basic Level (nota 5 || 6) :
\begin{itemize}
	\item Realizeaza o aplicatie simpla "Hello world" care va contine 2 butoane care vor afisa 2 pagini diferite, folosing 2 elemente diferite de interactiune
\end{itemize}

Normal Level (nota 7 || 8):
\begin{itemize}
	\item Implimenteaza un simplu ceas sau stopwatch
\end{itemize}

Advanced Level (nota 9 || 10):
\begin{itemize}
	\item Realizeaza o aplicatie care va implimenta tehnica Pomodoro SAU
	\item O alta aplicatie sofisticata la alegere (Game)
\end{itemize}
\subsection{Analiza lucrarii de laborator}

	Linkul repozitoriului  \url{https://github.com/artiomnichifor/midps}

	Scopul lucrarii date a fost cel de a forma cunostinte si perceptii despre functionarea unui sistem de operare pentru dispozitive mobile si formarea unei aplicatii pe aceasta platforma. Aici sunt reprezentate etapele formarii unei aplicatii Android in limbajul Java cu ajutorul mediului Android Studio.
	 
	Android este o platformă software și un sistem de operare pentru dispozitive și telefoane mobile bazată pe nucleul Linux, care permite dezvoltatorilor să scrie cod gestionat în limbajul Java, controlând dispozitivul prin intermediul bibliotecilor acestui limbaj dezvoltate de Google. SDK-ul (Software Development Kit) Android include un set complet de instrumente de dezvoltare. Acestea includ un program de depanare, biblioteci, un emulator de dispozitiv, documentație, mostre de cod și tutoriale. Platformele de dezvoltare sprijinite în prezent includ calculatoare bazate pe x86 care rulează Linux (orice distribuție Linux desktop modernă), Mac OS X 10.4.8 sau mai recent, Windows XP sau Vista. 
	
	Mediile de dezvoltare (IDE) suportat oficial de Google fiind Eclipse, iar apoi Android Studio. Android Studio reprezinta un mediu integrat de dezvoltare aparut recent, dovedit a fi foarte eficient in dezvoltarea aplicatiilor. Ide -ul dat are o multime de particularitati precum: 
\begin{itemize}
	\item functionarea dupa principiul WYSIWYG (What You See Is What You Get), posibilitatea de lucru cu elementele UI cu ajutorul functiei Dag-and-Drop
	\item Refactoring ul codului
	\item Analizator static Lint
	\item Sabloane integrate a unei aplicatii Android
	\item Formarea aplicatiilor pe baza Gradle etc.
\end{itemize}
	Aplicatia din continutul lucrarii date satisface cerintele laboratorului, efectuind urmatoarele etape:
\begin{itemize}
\item Realizeaza o aplicatie simpla "Hello world" care va contine 2 butoane care vor afisa 3 pagini diferite
\item Implimenteaza un simplu stopwatch cu trei butoane
\item Implementeaza un simplu joc "Drink or die" 
\end{itemize}
	
	Orice aplicatie Android se bazeaza pe principiile de interactiune dintre elementele sale sale precum ierarhiile de layouts, containere care conduc spre copii ramurei ierarhice, widgets, simple componente UI, activities, fiecare pagina de pe layout, intents, obiecte ce reflecta legatura dintre componentele precedente.


\subsection{Imagini}

\includegraphics[]{C:/Users/nichi/Desktop/Capture2.PNG}

	Main activity sau primul layout si utilizarea cimpului send message

\includegraphics[]{C:/Users/nichi/Desktop/Capture3.PNG}

	Deschiderea urmatoarei pagini si afisarea mesajului

\includegraphics[]{C:/Users/nichi/Desktop/Capture4.PNG}

	Functionarea stopwatch -ului dupa tastarea butonului tap me

\includegraphics[]{C:/Users/nichi/Desktop/Capture5.PNG}

	Functionarea jocului "Drink or die"






